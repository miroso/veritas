\chapter{Abstract}

In the mammalian retina, the photoreceptor signal is decomposed at the first synapse into multiple parallel channels, providing the foundation of the functional diversity of the retinal ganglion cells (RGCs). These neurons encode different features of the visual world into spike trains and send this information to different central targets. Systematic anatomical studies distinguished approximately 20 different RGC types based on their stratification depth, and the size and the shape of their dendritic field (e.g. \citealp{voelgyi09, kong05}). In a collaboration between the Bethge group and our lab, we are currently compiling a functional classification of mouse RGC output using population Ca$^{\text{2+}}$ imaging of RGC responses driven by a comprehensive set of visual stimuli and cluster analysis. Here, our aim is to complement this large-scale imaging approach by providing spiking activity and morphology of individual RGCs from the different functional clusters. 

We use cell-attached recordings in the whole-mounted mouse retina to monitor RGC spiking activity evoked by a set of stimuli (the same as is used in the aforementioned imaging approach), including frequency/contrast modulated full-field and spatial white noise to map receptive fields. Recorded cells are subsequently filled using Sulforhodamine 101 and imaged using a two-photon microscope to acquire their morphology, which is then reconstructed offline using Simple Neurite Tracer (\url{http://fiji.sc/Simple_Neurite_Tracer}). In some experiments, the retina is electroporated with synthetic Ca$^{\text{2+}}$  indicator \citep{briggman11} to allow targeting RGCs for electrical recordings by their Ca$^{\text{2+}}$  signal response. This way, integration of the single-cell data into the large-scale imaging data set can be greatly facilitated. Simultaneous voltage and Ca$^{\text{2+}}$  measurements also allow us to assay how spiking relates to somatic Ca$^{\text{2+}}$  changes in different types of RGC. 

In conclusion, this study, in combination with the large-scale classification project, will allow linking functionally defined RGC types to the morphological correlate – adhering the classical credo of retinal research that form follows function. It is an important step forward in our goal of capturing the complete set of the information the mouse eye sends to the mouse brain.  
 
