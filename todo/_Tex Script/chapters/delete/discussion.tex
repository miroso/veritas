\chapter{Discussion}

\section{Linking Form to Function}
There are many studies that tried to identify different RGC types, based solely on their morphological (e.g. \citealp{voelgyi09}) or physiological features (e.g. \citealp{farrow11}), but only a few were dedicated to establish a correlation between the morphology and the respective functional role of the cell \citep{pang03, weng05, berson08}. A comprehensive survey of mouse RGCs linking form to function is currently lacking. The methodology used in this study allows us to directly link the morphology of each individual cell with its physiological correlate. This dataset can then be quantitatively compared to already well-defined RGC types across different species.  

One of the most well-established cell types in the mammalian retina are the so-called “alpha cells”. They were originally described by \citet{boycott74} in the cat retina and can easily be recognized across different species by their large cell bodies, relatively stout dendrites, and large dendritic fields with minimal dendritic overlap. Physiologically, these cells exhibit transient light responses and high contrast sensitivity \citep{peichl81}. Three distinct Alpha-like cells have been described previously in the mouse retina. Two of these types form a paramorphic pair, termed RG$_{\text{A2}}$, that is arguably homologous to cat alpha cells. The third is an unpaired ON type, termed RG$_{\text{A1}}$ and characterized by a slightly larger, sparser dendritic profile that stratifies just beneath the presumed ON alpha homolog \citep{sun02, berson08}.\\
We suggested two cells to be equivalent to “alpha cells”. Both GC$_{\text{1M}}$ and GC$_{\text{2M}}$ stratified at the border between the OFF and the ON IPL lamina, and shared many physiological and morphological features (Fig.~\ref{figure1} and ~\ref{figure2}). However, GC$_{\text{2M}}$ showed a transient ON-OFF response profile and had in contrast to GC$_{\text{1M}}$ an anisotropic dendritic arbor with a displaced cell body, probably responsible for the strong direction selectivity. Despite these differences, both cells were classified previously \citep{kong05, voelgyi09} to belong to the same RGC cluster. 

Another RGC type that appears to be highly consistent across different mammalian species is a small-field cell with a dense dendritic arbor, mono-stratifying near the boundary between the ON and OFF laminas of the IPL, and having an ON-OFF receptive field center \citep{berson08}. This ganglion cell was initially described physiologically in the rabbit retina by \citet{Levick67} and based on its unique physiological properties it was termed a “local edge detector” (LED). In brief, LEDs exhibit a high preference for stimuli, which are restricted to their small receptive field center, and show only a poor responsiveness to full-field stimuli. Similarly to the ON-OFF DS cell (see below), LEDs respond equally well to small light and dark spots \citep{berson08}.  \\
The physiological and morphological properties of GC$_{\text{5M}}$ indicate that this cell could be the homolog of the rabbit LED in the mouse retina. In correspondence to the rabbit LED, GC$_{\text{5M}}$ has a small and dense dendritic arbor, ramifying at the border between the ON and OFF laminas of the IPL (Fig.~\ref{figure5}A and B), thus giving the cell an ON-OFF response profile. The analysis of the extracellular spikes resembled the physiological description of a LED. GC$_{\text{5M}}$ responded poorly to full-field stimuli (Fig.~\ref{figure5}E and F) and not at all to the fast full-field stimulation protocol (Fig.~\ref{figure5}D bottom). In comparison, the cell responded remarkably well to the DN - checkerboard - stimulus, yielding an ON-OFF receptive field (Fig.~\ref{figure5}C).\\
There are other candidates, which have previously been proposed to incorporate the role of a LED in the mouse retina \citep{berson08}. These are the B2, B4 and C5 types of \citet{sun02} and their morphological correlates in other RGC catalogues. However, these types are still lacking quantitative physiological descriptions and their role remains controversial. 

We identified one cell that appears to play an important role in response gain during contrast adaptation. The term “adaptation” defines a process where the gain of the response decreases to a constant feature of the stimulus. This is essential for the information processing in the visual system, because it averts saturation of the response to strong stimuli and permits for continued signaling of future increments of the stimulus strength \citep{laughlin89, smirnakis97}. On the other hand, adaptation has a detrimental effect on the sensitivity to a future decrease in stimulus strength. According to a recent study, the information loss in the retina is compensated by two different RGC populations - one that adapts following a strong stimulus and another that becomes sensitized - thus, providing an overall improvement of information processing by the retinal output neurons \citep{kastner11}.\\
The GC$_{\text{7M}}$ seems to resemble the behavior of the second group of RGCs that gradually become sensitized following a strong stimulus: The light-evoked action potentials of the GC$_{\text{7M}}$ become gradually suppressed to an increase in the stimulus strength during the contrast sweep,  and are followed by an enhanced sensitivity to light modulation underlying an increased spiking activity of the cell  (Fig.~\ref{figure7}E). 

Probably the best characterized RGC type across any mammalian species is the ON-OFF DS cell, which was first described functionally in the rabbit retina by \citeauthor{barlow63} in \citeyear{barlow63}. The cell can easily be recognized by its key-features, which are a clearly bi-stratified dendritic arbor, with an inner arbor stratifying in the OFF IPL lamina and an outer arbor in the ON lamina, and a prominent selectivity for the direction of the stimulus motion (for more information on DS, see \citealp{vaney12}).\\
We identified one cell that has homologous morphological features to the ON-OFF DS cell (Fig.~\ref{figure8}A and B), but in contrast the GC$_{\text{1B}}$ did not show any direction selective preference to the motion of bar stimuli (Fig.~\ref{figure8}G). However, the GC$_{\text{1B}}$ responded only sparsely to the full-field stimuli (Fig.~\ref{figure8}E and F) and it appears that not all extracellular spikes were locked to the light modulation, suggesting that some of the photoreceptors contributing to the physiological behavior of the cell might have been damaged during the preparation of the retina. 

The AC identified during the course of this study (Fig.~\ref{figure9}) shared many morphological features with the WA-S3 type, a displaced AC described previously by \citet{sevilla07}. Even though, the authors claim that WA-S3 was intentionally displaced to GCL and thus has no other counterpart located in the INL, \citet{lin06} described a wide-field amacrine cell, WA3-2, with its cell body located in the INL and similar morphological structure and stratification level as the WA-S3. According to \citet{lin06}, no other counterparts of WA3-2 have been reported in other species.\\
There is evidence that several WA3-2 form functional connections to only one sub-type of RGC. In a classification study, \citet{voelgyi09} injected Neurobiotin, a gap junction-permanent tracer, into a wild-type retina to determine the coupling patterns of cells located in the GCL. The resulting micrographs revealed that several WA3-2 ACs form electrical connections to only one RGC type (the G8 cell type described in the same study), suggesting that the WA3-2 cells contribute to the intrinsic mechanisms that account for the spiking activity of this RGC type.\\
In comparison to WA-S3, AC$_{\text{1M}}$ displayed only short dendrites whereas the dendritic arbor of WA-S3 should extend for several millimeters. Here, it is important to note that there is a high probability that the diffusion of SR-101 did not reach the distal endings of the dendrites and thus did not completely stain the whole morphology of the cell. On the other hand, AC$_{\text{1M}}$ could also represent a new cell type that has not been described yet by any previous studies. 

The remaining cells in this study did not exhibit any unique behavioral or morphological features, and were therefore not link to their functional role in the retina.

%%%%%%%%%%%%%%%%%%%%%%%%%%%%%%%%%%%%%%%%%%%%%%%%%%%%%%%%%%%%%%%%%%%%
\section{Towards a Complete Representation of Mouse RGCs}

\begin{figure}[t]
\begin{center}
\includegraphics[width=14cm]{images/simultrec.eps}
\caption{Linking single cell data to a large-scale functional classification of mouse RGC outputs using population Ca$^{\text{2+}}$. \textbf{A:} A micrograph showing a retina electroporated with a synthetic Ca$^{\text{2+}}$ indicator. The cell of interest (COI) is shown before (left) and after simultaneous voltage and calcium recordings (right). \textbf{B:} Functional clustering of distinct RGC types by mean fluorescence responses ($\Delta$F/F) to the Chirp stimulus; kindly provided by T. Baden and P. Berens. \textbf{C:} Extracellular (top) and normalized fluorescence response of the COI (bottom) to the Chirp stimulus; the faint red color indicates the raw response of the COI.}
\label{figure10}
\end{center}
\end{figure}

As a joint project between our lab and the Bethge lab, we are currently trying to assess a complete functional classification of mouse RGC responses using population calcium imaging. In this study, our aim was to compile a comprehensive dataset, comprising both physiological and morphological data of individual RGCs and displaced ACs in the mouse retina, and to establish a link between the single cell data and the large-scale functional classification of RGCs. Such a link would not only allow us to correlate the functionally defined RGC types with their respective morphological structure, but also to infer their functional role in the retina. 

To do so, we performed simultaneous voltage and calcium recordings on retinae, which were electroporated with a synthetic Ca$^{\text{2+}}$ indicator \citep{briggman11}. In an exemplary experiment (Fig.~\ref{figure10}), we measured the simultaneous activity of one cell to a chirp stimulus (Fig.~\ref{figure10}C). The mean calcium trace was then compared offline to distinct functional chirp clusters (Fig.~\ref{figure10}B; kindly provided by T. Baden and P. Berens).  

This methodological approach enables us to determine how the extracellular action potentials relate to somatic Ca$^{\text{2+}}$ changes across different types of RGCs, and thus allows to directly link the single cell data to the respective functional correlates. The simultaneous voltage and calcium imaging could also be combined online with the functional classification data to allow selective targeting of individual RGCs for electrical recordings by their somatic Ca$^{\text{2+}}$ activity. This approach would immensely improve the efficiency for obtaining anatomical and physiological data from rare RGC types or cells of interest, and moreover it would greatly facilitate the integration of single cell data into the large-scale functional RGC classification.\\
This is an important step towards a complete anatomical and functional classification of retinal output neurons and the set of the information the mouse eye sends to the mouse brain.  

