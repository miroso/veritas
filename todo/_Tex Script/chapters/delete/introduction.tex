\chapter{Introduction}

Vision is perhaps the most fundamental of our senses. Our eyes allow us to see the world we live in, to gain knowledge through observing and reading, and they play an important role in communication with other people. The visual scene is focussed through the optics of the eyeball (for more details see \citealp{schaffel06}) onto a light-sensitive neural computer, called the retina \citep{masland01}. Here, the images from the outside world are being processed and projected via output neurons to distinct visual centers in the brain. 

The retina comprises 5 cardinal cell classes, which are organized in a layered structure, with each layer containing either cell bodies (nuclear layers) or synaptic neuropil (plexiform layers). At the input stage, local brightness changes over time are transduced into electrical signal by rod and cone-type photoreceptors, which are located in the outermost layer of the retina. 

In the outer plexiform layer (OPL) the light-evoked changes in the membrane potential of photoreceptors are synaptically conveyed to horizontal and bipolar cells (BCs). Horizontal cells (HCs) provide lateral inhibition and support gain control in the retina \citep{wassle04}, whereas bipolar cells split the photoreceptor signal into parallel channels \citep{wassle04, masland12, baden13} and relay the signal into the inner plexiform layer (IPL), where they form connections with the dendrites of amacrine (ACs) and RGCs \citep{wassle04}. Types of BCs most notably vary in their polarity and temporal properties, and systematically stratify at different depths of the IPL \citep{roska01, baden13}. 

There are approximately 30-40 different types of AC in the vertebrate retina, which vary widely in morphology and function \citep{strettoi96, macneil98}. Some of these cells contain Na$^{\text{2+}}$-gated voltage channels and can fire action potentials during  depolarization \citep{heflin07}. ACs can be divided based on their dendritic field size into two morphologically distinct groups.\\
Wide-field amacrine cells usually exhibit large dendritic arbors (>200 \textmu m in diameter) that generally stratifies only in a few sublaminae of the IPL. In contrast, narrow-field ACs have small dendritic arbors (<200 \textmu m in diameter), ramifying through several sublaminae \citep{roska01}. 
The cell bodies of ACs can be located either in the inner nuclear layer (INL) or in the GCL. The latter are termed “displaced amacrine cells” and so far only 17 distinct subtypes have been identified in the mouse retina by morphology \citep{lin06, sevilla07}. They can easily be differentiated from the ganglion cells, because of the small size of their cell body and the lack of an axon projecting into the brain.  

RGCs are the output neurons of the retina. Their cell bodies form a substantial part of the innermost layer. The functional diversity of these cells is formed by inhibitory inputs from various types of ACs \citep{oelveczky03, wassle04} and by the signal from parallel BC channels \citep{baden13}. Different RGC types are selectively tuned to different visual features, including spatio-temporal properties, chromaticity and overall light intensity \citep{gollisch09}. The segregated visual information is then encoded into spike-trains and projected via the optic nerve to specific visual centers in the brain. 

RGCs can be distinguished on the basis of morphological and physiological criteria (reviewed in \citealp{berson08}). The modern anatomical taxonomy was established in early studies using Golgi, Nissl, and neurofibrillar stains (\citealp{stone83}), and was based on different morphometric parameters, including dendritic morphology, size of the cell body and the dendritic tree, and the stratification depth in the IPL. The classification of RGCs advanced rapidly with the introduction of methods for intracellular dye filling, initially achieved through intracellular sharp electrodes in fixed or living tissue (e.g. \citealp{buhl86, pu90}). The breakthrough came with the approach of modern bio-molecular techniques using large-scale chemical mutagenesis and targeted genetic manipulation. In particular transgenic mouse strains, in which marker proteins are expressed under the control of cell-type-specific promoters, have been used with great success to systematically characterize retinal cell types \citep{badea04, coombs06, kong05}. Next to the mouse, the characterization of RGCs progressed most in rabbit \citep{rockhill02}, cat \citep{brien02}, rat \citep{huxlin97} and primates \citep{dacey94, yamada96}. 

The identification of functional cell types is essential for understanding the underlying computational circuitry of the RGCs. For a long time, the physiological classification of RGCs was based solely on single-unit studies, with recordings made \emph{in vivo} either intra-ocularly \citep{barlow64} directly from their cell bodies or intracranially from their axons in the optic tract \citep{enroth66}. \emph{In vitro} methods together with the development of optical and electrophysiological population recordings, such as calcium imaging or microelectrode arrays (MEAs), greatly increased the rate, at which single and relatively rare RGC types were functionally classified \citep{devries97}. 

Probably the most complete morphological classification of retinal output neurons has been achieved by \citet{voelgyi09} who were able to distinguish 22 different RGC subtypes in the mouse retina. In comparison, \citet{sun02} identified 17, \citet{kong05} 11, \citet{badea04} 12 and \citet{coombs06} 19 distinct RGC types or clusters. This large disparity across these morphological surveys could be perhaps attributed to different labeling and analyzing techniques, but also to the fact that some studies incorporated multiple RGC types into one cluster. This ambiguity could be resolved by linking anatomical structure to its appropriate physiological correlate. A possible solution for this ambiguity would be to link the anatomical structure directly to its physiological correlate. 

The goal of this study was to provide combined anatomical and physiological data for retinal output neurons in the whole-mounted mouse retina. In collaboration between our lab and Bethge group, we plan to use this dataset to complement a large-scale functional classification of retinal output neurons, compiled of population Ca$^{\text{2+}}$ imaging of RGC responses driven by the same set of visual stimuli, and cluster analysis. This approach would bring us closer to obtain a complete catalogue of functionally defined RGC types that could be linked to their morphological correlates.





