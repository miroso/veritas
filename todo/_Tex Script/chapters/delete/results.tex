\chapter{Results}

The goal of this study was to determine physiological and morphological features of cells located in the ganglion cell layer of the mouse retina. In total, light-evoked action potentials were recorded extracellularly from 30 cells. However, only 14 cells were successfully stained with SR-101. The analysis discarded another 5 cells due to poor signal-to-noise ratio, thus resulting into a complete dataset of 9 cells. In addition, simultaneous voltage and calcium recordings were successfully performed for 1 cell. On average, three cells have been obtained from one retina. 

The distinct physiological features of each cell were examined using a comprehensive set of stimuli: The type (ON, OFF, ON-OFF) of the cell and the spatiotemporal properties of the receptive field were mapped with spatial white noise stimuli. The chirp stimulus has been used to distinguish cells based on their response characteristics (sustained or transient) and their responses to divergent frequency and contrast modulation sweeps. The cell’s chromatic preference was determined by the BG stimulus. The direction selective stimulus tested the cells for DS responses by presenting moving bars in 8 different directions. 

To identify each cell stained with SR-101, an elaborate morphometric analysis has been carried out, including following parameters: 1) shape of the cell body; 2) dendritic field size; 3) stratification level of the dendritic arbor. Based on these criteria, we differentiated 7 mono-stratified and 1 bi-stratified ganglion cells, and 1 displaced AC. The cells were then related to cell types described previously, \citep{badea04, coombs06, sun02, kong05, voelgyi09} for ganglion cells and \citep{sevilla07} for displaced ACs. 


\section{Mono-stratified Ganglion Cells}
%%%%%%%%%%%%%%%%%%%%%%%%%%%%%%%%%%%%%%%%%%%%%%%%%%%%%%%%%%%%%%%%%%%%%%%%%
\subsection{GC 1M} 

The cell has a relatively large spherical cell body of 21 \textmu m in diameter, with three stout and smooth dendrites stratifying at 35\% in the second sublamina of the IPL (Fig.~\ref{figure1}A and B). The dendrites radiated from the soma and branched at regular intervals and formed a dendritic arbor of 198 \textmu m in diameter with no overlap. The dendritic arbor was slightly bent, causing the dendritic endings to protrude into the ON lamina of the IPL. The size of the cell body and the morphological structure is homologous to the A2 (outer) cell described by \citep{sun02} and cells in cluster 7 \citep{kong05}. In addition, GC$_{\text{1M}}$\footnote{M refers to mono-stratified} shares morphological features with cells defined as cluster 7 \citep{badea04}, M9 (off) \citep{coombs06} and G3 \citep{voelgyi09}. 

\begin{figure}[!t]
\begin{center}
\includegraphics[width=16cm]{images/GC1M.eps}
\caption{Morphology and physiology of a mono-stratified ganglion cell subtype (GC$_\text{{1M}}$) in the mouse retina. \textbf{A:} Photomicrograph showing the soma and the dendritic architecture (top), and the stratification in the IPL (bottom). \textbf{B:} Reconstructed cell morphology displaying the soma and the dendritic arbor (top), and the stratification depth across 5 IPL sublaminas (bottom). The arrows mark the axon. The name of the cell and its defining parameters – size of the soma and the dendritic arbor, and stratification depth – are included. \textbf{C:}  Reverse correlation of the cell’s response to a checkerboard stimulus, showing a spatial ON (top) and OFF receptive field (middle), and the standard deviation (bottom). (1 pixel = 40 \textmu m). \textbf{D:} Temporal receptive field filter calculated from a response to a checkerboard (top) and from a fast full field stimulus (bottom). The bi-phasic index (Bi) determines the bi-phasicity of the linear filter. \textbf{E:} Responses to a full-field stimulus, including light steps, frequency- and contrast-sweeps. \textbf{F:} Chromatic preference of the cell, computed from a response to "green" and "blue" light steps. The blue-green index (BGi) determines the chromatic preference, with positive values indicating "green" preference and negative values “blue” preference. \textbf{G:} Direction selectivity determined from responses to bars moving in different angular directions.}
\label{figure1}
\end{center}
\end{figure}

GC$_{\text{1M}}$ showed relatively large receptive field (Fig.~\ref{figure1}C). The negative polarity of the main lobe in the linear filter (Fig.~\ref{figure1}D) and the clear transient OFF response to a light step (Fig.~\ref{figure1}E), and the BG stimulus (Fig.~\ref{figure1}F) indicate that GC$_{\text{1M}}$ is a type OFF cell. Furthermore, GC$_{\text{1M}}$ was very insensitive to high frequency stimulation, but responded in a phase-locked manner to all but the onset of the contrast sweep (Fig.~\ref{figure1}E). GC$_{\text{1M}}$ displays a slight preference for the green component of the dichromatic stimulation (Fig.~\ref{figure1}F). The light-evoked responses to bars moving in 8 angular directions were mostly evenly distributed and therefore did not reveal any preferred direction selectivity (Fig.~\ref{figure1}G). 
 
 
%%%%%%%%%%%%%%%%%%%%%%%%%%%%%%%%%%%%%%%%%%%%%%%%%%%%%%%%%%%%%%%%%%%%%%%%%
\subsection{GC 2M}     

\begin{figure}[t]
\begin{center}
\includegraphics[width=16cm]{images/GC2M.eps}
\caption{Morphology and physiology of a mono-stratified ganglion cell subtype (GC$_\text{{2M}}$) in the mouse retina. Conventions as in \textbf{Figure 1}.}
\label{figure2}
\end{center}
\end{figure}

The cell displays a relatively large elliptical cell body of 20 \textmu m in diameter (Fig.~\ref{figure2}A and B). Four thick primary dendrites extended for several micrometers before they ramified narrowly between the second and the third IPL sublamina at a depth of 42\%. The dendritic arbor with 222 \textmu m in diameter had a slight anisotropy, with dendritic branches directed to one side of the cell body. GC$_{\text{2M}}$ shares many morphological features with GC$_{\text{1M}}$ cell and according to other morphological surveys, GC$_{\text{2M}}$ belongs to the same cluster as GC$_{\text{1M}}$ \citep{kong05, voelgyi09}. 

The white noise analysis revealed a relatively large elliptical receptive field (Fig.~\ref{figure2}C) and a linear filter that characterized the cell to be an OFF type (Fig.~\ref{figure2}D). Strikingly, the cell displayed a transient ON-OFF response profile to a light step (Fig.~\ref{figure2}E) and the blue-green stimulation (Fig.~\ref{figure2}F). This can probably be attributed to the large dendrites, which protrude the ON lamina of the IPL and stratify in the OFF lamina. Furthermore, GC$_{\text{2M}}$ was phase-locked to all but high frequency and low contrast modulation (Fig.~\ref{figure2}E). The analysis of the BG stimulus revealed a slight preference for the blue component (Fig.~\ref{figure2}F) and the DS analysis rendered the cell strongly direction selective (Fig.~\ref{figure2}G), probably induced by the anisotropy of the dendritic arbor and the displaced cell body. 


%%%%%%%%%%%%%%%%%%%%%%%%%%%%%%%%%%%%%%%%%%%%%%%%%%%%%%%%%%%%%%%%%%%%%%%%%
\subsection{GC 3M}     

\begin{figure}[t]
\begin{center}
\includegraphics[width=16cm]{images/GC3M.eps}
\caption{Morphology and physiology of a mono-stratified ganglion cell subtype (GC$_\text{{3M}}$) in the mouse retina. Conventions as in \textbf{Figure 1}.}
\label{figure3}
\end{center}
\end{figure}

GC$_{\text{3M}}$ shows an elliptical cell body of medium size, 17 \textmu m in diameter, and five primary dendrites stratifying narrowly at a depth of 86\% in the fifth IPL sublamina (Fig.~\ref{figure3}A and B). In general, the dendrites were very thin, branched at acute angles and stayed in close proximity to each other. The dendritic arbor was of medium size with a diameter of 193 \textmu m and showed a slight dendritic overlap. The size of the soma and the architecture of the dendritic arbors are nearly identical to B3 (inner) \citep{sun02}, cluster 4 \citep{kong05}, cluster 9 \citep{badea04}, M3 (on) \citep{coombs06} and to G6 cells \citep{voelgyi09}. 

The DN analysis revealed a round receptive field of approximately 200 \textmu m in diameter (Fig.~\ref{figure3}C). The linear filter classified this cell to be ON type (Fig.~\ref{figure3}D). GC$_{\text{3M}}$  displayed a sustained ON response and a phase-locked sensitivity to higher contrast stimulation, but no clear preference for the frequency modulation (Fig.~\ref{figure3}E). GC$_{\text{3M}}$ showed no preference to the dichromatic stimulation (Fig.~\ref{figure3}F) as well as to the DS stimulus (Fig.~\ref{figure3}G).


%%%%%%%%%%%%%%%%%%%%%%%%%%%%%%%%%%%%%%%%%%%%%%%%%%%%%%%%%%%%%%%%%%%%%%%%%
\subsection{GC 4M}     

\begin{figure}[t]
\begin{center}
\includegraphics[width=16cm]{images/GC4M.eps}
\caption{Morphology and physiology of a mono-stratified ganglion cell subtype (GC$_\text{{4M}}$) in the mouse retina. Conventions as in \textbf{Figure 1}.}
\label{figure4}
\end{center}
\end{figure}

GC$_{\text{4M}}$ has a round cell body of 16 \textmu m in diameter with three primary dendrites arborizing at the border between the fourth and fifth IPL sublamina at a depth of 78\% (Fig.~\ref{figure4}A and B). Four primary dendrites of medium thickness projected from the soma and ran for several micrometers before they bifurcated. The dendrites branched sparsely with only a little overlap. The dendritic ramifications formed a large dendritic arbor of 290 \textmu m. GC$_{\text{4M}}$ resembles cells described as C2 (inner) \citep{sun02}, cluster 5 \citep{kong05}, M10 \citep{coombs06} and G10 \citep{voelgyi09}. There was no obvious counterpart in \citet{badea04}.  

The GC$_{\text{4M}}$ did not respond to DN stimulus and therefore no linear filter and receptive field could be calculated (Fig.~\ref{figure4}C and D). The cell showed a slow sustained ON response to light steps and reacted poorly to frequency and contrast modulation (Fig.~\ref{figure4}E). The analysis of the 
BG stimulus revealed chromatic preference for the blue component (Fig.~\ref{figure4}F). Moreover, GC$_{\text{4M}}$ appears to be direction selective (Fig.~\ref{figure4}G). 


%%%%%%%%%%%%%%%%%%%%%%%%%%%%%%%%%%%%%%%%%%%%%%%%%%%%%%%%%%%%%%%%%%%%%%%%%
\subsection{GC 5M}     

\begin{figure}[t]
\begin{center}
\includegraphics[width=16cm]{images/GC5M.eps}
\caption{Morphology and physiology of a mono-stratified ganglion cell subtype (GC$_\text{{5M}}$) in the mouse retina. Conventions as in \textbf{Figure 1}.}
\label{figure5}
\end{center}
\end{figure}

GC$_{\text{5M}}$ shows a round cell body of medium size with 15 \textmu m in diameter and five primary dendrites, branching immediately next to the soma (Fig.~\ref{figure5}A and B). The dendrites ramified repetitively in close proximity to each other and diffused through several IPL sublamina, stratifying at a depth between 9\% and 67\%. Thus, the dendrites formed a rather small and dense dendritic arbor of 142 \textmu m in diameter with a high dendritic overlap. These distinct morphological features and stratification level are identical to C4 cell \citep{sun02}, cluster 1 \citep{kong05}, mono-stratified cluster 4 \citep{badea04}, M11 \citep{coombs06} and to G13 ganglion cell described by \citet{voelgyi09}. 

GC$_{\text{5M}}$ cell has a small receptive field of approximately 150 \textmu m in diameter (Fig.~\ref{figure5}C), which is consistent with morphometric analysis of the cell’s dendritic arbor. The linear filter indicates that the cell is an OFF type (Fig.~\ref{figure5}D). However, the cell responded to both the onset and the offset of a light step (Fig.~\ref{figure5}E) and the dichromatic stimulation (Fig.~\ref{figure5}F). This ON-OFF resemblance can probably be attributed to the thick dendritic arbor, protruding the OFF as well as ON IPL laminae. Moreover, GC$_{\text{5M}}$ reacted only sparsely to the frequency and contrast modulation (Fig.~\ref{figure5}E). The stimulation protocol for the chromatic preference did no show any significant differences between the dichromatic stimulation (Fig.~\ref{figure5}F). On the other side, the DS stimulus elicited a clear direction selective preference  (Fig.~\ref{figure5}G). 


%%%%%%%%%%%%%%%%%%%%%%%%%%%%%%%%%%%%%%%%%%%%%%%%%%%%%%%%%%%%%%%%%%%%%%%%%
\subsection{GC 6M}     

\begin{figure}[t]
\begin{center}
\includegraphics[width=16cm]{images/GC6M.eps}
\caption{Morphology and physiology of a mono-stratified ganglion cell subtype (GC$_\text{{6M}}$) in the mouse retina. Conventions as in \textbf{Figure 1}.}
\label{figure6}
\end{center}
\end{figure}

GC$_{\text{6M}}$ cell presents a round cell body of 15 \textmu m in diameter with one thick dendrite, extending for several micrometers and stratifying into 3 main branches at a depth between 5\% and 51\% (Fig.~\ref{figure6}A and B). The dendritic arbors are curved, overlapping each other, but generally project in radial direction. The dendritic field with 143 \textmu m is rather of smaller size. Cell 5 shares morphological features with C5 cell \citep{sun02}, cells in cluster 3 \citep{kong05}, cells in mono-stratified cluster 1 and 2 \citep{badea04}, cell M3 (off) \citep{coombs06} and with a cell G14 \citep{voelgyi09}. 

The small size of the dendritic arbor was reflected in the size of the receptive field as a result of the DN analysis (Fig.~\ref{figure6}C). GC$_{\text{6M}}$ can be classified as a transient OFF type, as indicated by the linear filter (Fig.~\ref{figure6}D) and the light-evoked responses to the light step (Fig.~\ref{figure6}E) and the dichromatic stimulation (Fig.~\ref{figure6}F). Further, the cell was insensitive to high frequency modulation and low contrast changes (Fig.~\ref{figure6}E). The BG and DS analysis did not show any preferences (Fig.~\ref{figure6}F and G, respectively).  


%%%%%%%%%%%%%%%%%%%%%%%%%%%%%%%%%%%%%%%%%%%%%%%%%%%%%%%%%%%%%%%%%%%%%%%%%
\subsection{GC 7M}     

\begin{figure}[t]
\begin{center}
\includegraphics[width=16cm]{images/GC7M.eps}
\caption{Morphology and physiology of a mono-stratified ganglion cell subtype (GC$_\text{{7M}}$) in the mouse retina. Conventions as in \textbf{Figure 1}.}
\label{figure7}
\end{center}
\end{figure}

GC$_{\text{7M}}$ cell has a small elliptical cell body of 14 \textmu m in diameter with five thin dendrites stratifying between the third and fourth IPL sublamina at a depth of 59\% (Fig.~\ref{figure7}A and B). The dendrites radiated from the soma and branched only sparsely with merely no overlap. The dendritic arbor of 217 \textmu m in diameter was slightly bent with dendritic endings protruding to the GCL. The dendritic architecture of GC$_{\text{7M}}$ resembles somewhat descriptions of cells defined as C2 (inner) cells \citep{sun02}, mono-stratified cluster 9 \citep{badea04}, cluster 9 \citep{kong05}, M6 (on) \citep{coombs06} and G10, which were described by \citet{voelgyi09}. 

GC$_{\text{7M}}$ is a type ON cell (Fig.~\ref{figure7}D) with a receptive field of medium size and approximately 200 \textmu m (Fig.~\ref{figure7}C). The sustained temporal profile of the cell is reflected in the light-evoked responses to the Chirp (Fig.~\ref{figure7}E) and the BG stimulus (Fig.~\ref{figure7}F). GC$_{\text{7M}}$ responded with increasing firing rate to higher frequency modulation and with suppressed activity to higher contrast stimulation (Fig.~\ref{figure7}E). The BG and DS analysis did not reveal any chromatic (Fig.~\ref{figure7}F) or direction selective (Fig.~\ref{figure7}G) preferences, respectively.

\section{Bi-stratified Ganglion Cell}
%%%%%%%%%%%%%%%%%%%%%%%%%%%%%%%%%%%%%%%%%%%%%%%%%%%%%%%%%%%%%%%%%%%%%%%%%
\subsection{GC 1B} 

\begin{figure}[t]
\begin{center}
\includegraphics[width=16cm]{images/GC1B.eps}
\caption{Morphology and physiology of a bi-stratified ganglion cell subtype (GC$_\text{{1B}}$) in the mouse retina. Conventions as in \textbf{Figure 1}.}
\label{figure8}
\end{center}
\end{figure}

GC$_{\text{1B}}$\footnote{B refers to bi-stratified} displays an elliptical cell body of medium size and 18.5 \textmu m in diameter. The cell has four primary dendrites and bi-stratifies at a depth of 25\% and 61\% in the OFF and ON lamina of the IPL, respectively (Fig.~\ref{figure8}A and B). The dendritic arbor located in the ON lamina has a diameter of 168 \textmu m and an elongated shape with sparse dendritic ramifications. In comparison, dendritic field in the OFF lamina has a diameter of 184 \textmu m and a rather circular shape with more densely branched dendrites. GC$_{\text{1B}}$ is very similar to the D2 cell described by \citealp{sun02}, the G17 ganglion cell \citep{voelgyi09} and to the cells in the M13 cluster \citep{coombs06}. Cell 6 also shares several morphological features with the bi-stratified cluster 2 and 3 \citep{badea04}. \citet{kong05} did not classify any bi-stratified cells. 

The bi-stratified GC$_{\text{1B}}$ has a relatively large receptive field of 200 \textmu m (Fig.~\ref{figure8}C). The negative polarity of the main lobe of the linear filter, indicates stronger temporal OFF component (Fig.~\ref{figure8}D). The cell displays a transient response profile in the BG stimulus (Fig.~\ref{figure8}F) and responded somewhat to the onset and offset of a light step in the chirp stimulus (Fig.~\ref{figure8}E). The BG and DS stimulus analysis did not reveal any chromatic (Fig.~\ref{figure8}F) nor direction selective (Fig.~\ref{figure8}G) preferences, respectively.


\section{Displaced Amacrine Cell}
%%%%%%%%%%%%%%%%%%%%%%%%%%%%%%%%%%%%%%%%%%%%%%%%%%%%%%%%%%%%%%%%%%%%%%%%%
\subsection{AC 1M}     

\begin{figure}[t]
\begin{center}
\includegraphics[width=16cm]{images/AC1M.eps}
\caption{Morphology and physiology of a mono-stratified amacrine cell subtype (AC$_\text{{1M}}$) in the mouse retina. Conventions as in \textbf{Figure 1}.}
\label{figure9}
\end{center}
\end{figure}

The cell has a round soma of approximately 10 \textmu m in diameter and no axon. Its dendrites are narrowly mono-stratified at a depth of 45\% in the third IPL sublamina (Fig.~\ref{figure9}A and B). Two primary dendrites bifurcate several times in close proximity to the soma, resulting into a star-like appearance with 15 dendrites in total and a dendritic field of 253 \textmu m in diameter. The dendrites project radially away from the soma, getting thinner towards the distal endings and rarely crossing each other. The unambiguous appearance of the dendritic field resembles a characteristic shape of a wide-field amacrine cell. Such a displaced amacrine cell, WA-S3, was catalogued earlier by \citet{sevilla07} and further examined by \citet{majumdar09}. 

The DN analysis revealed a large receptive field of 200 \textmu m in diameter (Fig.~\ref{figure9}C) and a linear filter, indicating the cell to be an OFF type (Fig.~\ref{figure9}D). AC$_{\text{1M}}$ displays a transient response profile, as depicted by the Chirp and BG stimuli, and is phase-locked to high and medium frequency modulation and to the whole sweep of the sinusoidal contrast stimulation (Fig.~\ref{figure9}E). AC$_{\text{1M}}$ shows a slight preference for the blue component of the BG stimulus (Fig.~\ref{figure9}F). The light-evoked responses to bars moving in 8 angular directions were mostly evenly distributed across all directions and therefore did not revealed any preferred direction selectivity (Fig.~\ref{figure9}G).



