\chapter{Materials and Methods}
\subsection{Animals and Tissue Preparation}
All experiments were performed on male and female C57BL/6 mice, and all experimental procedures were carried out in accordance with the standards of the Society for Neuroscience and the law of animal experimentation issued by the German Federal Government (Tierschutzgesetz). 

The animals were dark-adapted for at least 60 minutes prior to the experiment, and all subsequent procedures were carried out under dim red light illumination. The mice were anesthetized with isoflurane inhalation (Baxter, Unterschleißheim, Germany) and sacrificed by cervical dislocation. Subsequently, the eyes were enucleated and transferred into a petri dish, filled with carboxygenated (5\% CO$_{2}$, 95\% O$_{2}$) ringer solution (in g) 14.61 NaCl, 0.373 KCl, 0.41 MgCl$_{2}$.6H$_{2}$O, 0.3 NaH$_{2}$PO$_{4}$, 7.2 L-glucose, 4.369 NaHCO$_{3}$, 0.588 CaCl$_{2}$ and 0.146 L-Glutamine in 2 l distilled H$_{2}$O) at room temperature. After the removal of the lens and cornea, the retina was separated from the pigment epithelium and cleaned from the vitreous humor. Finally, the tissue was equally cut at four cardinal directions, allowing the retina to be flat-mounted with photoreceptor-side down onto an anopore membrane (anodisc 13, 0.1 \textmu m pore size, Whatman GmbH, Dassel, Germany).

\subsection{Bulk Electroporation}
The procedure for bulk electroporation is described elsewhere \citep{briggman11}. In brief, the retina mounted on the anodisc was centered over a platinum electrode covered with 15 \textmu l of the ringer solution. Another platinum electrode with 10 \textmu l drop of 3.15 mM solution of the synthetic calcium indicator Oregon Green 488 BAPTA-1 (OGB-1; hexapotassium salt; Invitrogen, Darmstadt, Germany) was applied from above and lowered onto the retina until it formed a stable meniscus. The tissue was electroporated with 10 pulses (10 ms-pulse-width, 1 Hz) at +9 V and, using a pulse generator/wide-band amplifier combination (TGP110 and WA301, Thurlby Thandar/Farnell, Oberhaching, Germany). After the electroporation, the preparation was immediately transferred to the recording chamber, where it was held in place by a platinum weight and continuously perfused with carboxygenated ringer solution at 37°C. The tissue was left in the recording chamber for approximately 60 minutes to recover from the electroporation.  

\subsection{Electrophysiology and Cell Morphology}
After the tissue preparation, the retina was placed directly into the recording chamber where it was perfused with carboxygenated ringer solution at 37°C. For the electrophysiological recordings, glass micropipettes (8-10 M$\Omega$) were pulled from borosilicate glass capillaries (IB150F-4; World Precision Instruments, Sarasota, FL) with a P-1000 Flaming/Brown micropipette puller (Sutter Instruments, Novato, CA). The pipette was filled with (in mM) 135 NaCL, 5.4 KCl, 1.8 CaCl$_{2}$, 1 MgCl$_{2}$ and 5 HEPES. The pipette solution was adjusted to pH 7.2 with NaOH and contained sulforhodamine 101 (SR-101, 200 \textmu M, Sigma) to outline the cellular morphologies.\\
The pipette was lowered under the guidance of an infrared CCD camera (DCU223M, Thorlabs, uc480 Viewer software) until it gently pressed against the retina’s inner limiting membrane, which covers the ganglion cell layer. The electrode was then moved forward to pierce through the membrane, and once the micropipette was located under the membrane positive pressure was applied to reveal the contour of nearby cells. Extracellular spikes were recorded in the cell-attached mode using a Multiclamp amplifier (Axoclamp 900A and pClamp10 software, Molecular Devices, Berkshire, UK). The data was digitized at 10 kHz, high-pass filtered at 3 Hz and subsequently analyzed off-line in Igor Pro (Wavemetrics, Portland, OR, USA). 

After the electrophysiological recordings, the cell membrane was penetrated by the application of short current pulses (15 nA, 200 ms), complemented by a slow current injection to maintain the membrane potential at -10 mV (time constant: 20 ms). This allowed to reveal the recorded cell’s morphology with SR-101, which was subsequently captured as a series of images (512x512 pixels) recorded in 1 \textmu m z-steps on the two-photon microscope.

\subsection{Two-photon Imaging}
We used two-photon custom-build microscope (designed by W. Denk, Max Planck Institute for Medical Research [MPImF], Heidelberg, Germany; purchased from Sutter Instruments). Both design and procedures were described previously \citep{euler09, baden13}. In brief, the microscope was equipped with a mode-locked Ti/sapphire laser (MaiTai, Spectra Physics, Newport, Santa Clara, CA) tuned to 927 nm. The microscope was equipped with through-the-objective light stimulation and two detection channels for fluorescence imaging (green, 535 BP-50 and red, 650 BP-60, Chroma, Tübingen, Germany). The green channel was used to visualize OGB-1 fluorescence changes, while the red channel outlined the morphological structure of the recorded cell, stained with SR-101. 

Scans were recorded with a custom written software package “ScanM” (written by M. Müller, MPI for Medical Research, Heidelberg, and Thomas Euler, CIN, Tübingen), running in IgorPro 6.2 (Wavemetrics, Lake Oswego, OR, USA). Activity stacks were recorded at a resolution of 64 x 64 pixels with 2 ms per line (7.8 Hz frame rate), and contained a synchronization signal at sub-millisecond precision from the light stimulator, allowing precise temporal alignment of the calcium signal with the stimulus.

\subsection{Light Stimulation}
Light stimuli were projected onto the retinal photoreceptors via a light stimulator, based on a miniature DLP (Digital Light Processing) projector (Acer K11), of which the built-in illumination LEDs (light emitting diodes) have been exchanged by two band-pass-filtered (blue: 400 BP 20, green: 578 BP 10; AHF/Chroma) custom-selected LEDs. The DLP projector was coupled directly into the main optical path of the microscope, allowing the stimuli to be projected onto the flat-mounted retina through the objective lens. The light stimulator was driven by custom written software, running on a PC (Windows XP, Microsoft). To determine the various functional properties of each recorded cell, five different light stimuli protocols were presented:

\begin{enumerate}
\item Dense noise (DN) stimulus; 320 s stim. duration, e.g. (Fig.~\ref{figure1}C and D top). A checkerboard stimulus (40 \textmu m pixel length, 400 \textmu m x 300 \textmu m, 5 Hz, 100\% contrast) was used to determine the spatio-temporal properties. The stimulus consisted of square regions, each of which was modulated in time by a random binary signal. 
\item Fast full field (FF) stimulus; 270 s stim. duration, e.g. (Fig.~\ref{figure1}D bottom). A full field binary M-sequence was presented at 59 Hz (100\% contrast), to estimate the temporal response properties (“temporal linear filter”) at higher temporal precision than possible with the slower DN stimulus. 
\item Chirp stimulus; 5 repeats, e.g. (Fig.~\ref{figure1}E).  Sensitivity to different components of full field stimulation was studied with a stimulus composed of steps, frequency-sweep (linearly ramping from 0 to 10 Hz, sinusoidal, 100\% contrast) and contrast-sweep (2 Hz, sinusoidal, contrast linearly ramping from 0\% to 100\%).
\item Blue-Green (BG) stimulus; 7 repeats, e.g. (Fig.~\ref{figure1}F).  3 seconds “green” and “blue” steps (100\% contrast) were applied to test for chromatic preference. 
\item Direction selective (DS) stimulus; 3 repeats, e.g. (Fig.~\ref{figure1}G). A bright bar (300 \textmu m x 1000 \textmu m, 100\% and 50\% contrast) on a dark background moving in eight angular directions at 0.5 mm/s was used to test directional selectivity of the cell.  
\end{enumerate}

\subsection{Data Analysis}
The analysis for electrophysiological and calcium imaging data was carried out in IgorPro 6.2 (Wavemetrics, Lake Oswego, OR, USA) using custom-written procedures, based on scripts written by T. Baden (Euler lab). Fluorescence changes over time, reflecting the calcium signal, were extracted with the freely available image processing toolbox SAFRIA \citep{dorostkar10} and calculated as the ratio of the relative fluorescence change and the baseline fluorescence ($\Delta F/F$). The light-evoked responses to individual stimulus presentations were extracted using the synchronization signal. 

For the DN stimulus, the linear space and time receptive fields for each cell were estimated by reverse correlating the stimulus and response of the cell \citep{chichilnisky01, wang11}. The time course of the linear filter at the center of the receptive field indicates the polarity of the cell. A bi-phasic index (Bi) was defined as $B_{i}=\frac{P_{off}}{P_{on}}$, with $P_{off}$ referring to the peak of the OFF-lobe and $P_{off}$ referring to the peak of the ON-lobe \citep{chander01}. The FF stimulus was computed in a similar way as the DN stimulus.\\
Light-evoked responses for the chirp stimulus were averaged over trials. \\
The chromatic preference (Blue-Green index, BGi) was calculated in a following manner: $BG_{i}=\frac{G_{r}-B_{r}}{ G_{r}+B_{r}}$, with $BG_{i} > 0$ representing a green preference and $BG_{i} < 0$ defining a blue chromatic preference. $G_{r}$ refers to light-evoked responses by a green stimulus and $B_{r}$ to light-evoked responses by a blue stimulus.\\
To quantify extracellular spikes to the moving bar stimulus (DS stimulus), spike average over trials was calculated for each bar direction, resulting into 8 different response vectors (\emph{v}). A direction selective index (DSi) was indicated by the mean vector, with $DSi > 0.4$ indicating a directional tuning of the respective cell. The DS threshold of 0.4 was in accordance with previous measurements of DS in the mouse retina (e.g. \citealp{auferkorte12, rivlinetzion11}).

\subsection{Cell Reconstruction}
Cell morphology was reconstructed from the recorded image stacks using Fiji \citep{cardona12} and the simple neurite tracer plugin \citep{longair11}. The plugin also enabled volumetric rendering of the soma and the dendritic tree. Final micrographs were acquired using the volume rendering software, Voxx \citep{clendenon02}. The micrographs were then color-inverted in Fiji. \\
Based on reconstructed morphologies, the following parameters were quantified: (i) Soma diameter: The diameter of the cell body was computed in a horizontal plane from the long and short axis. (ii) Dendritic field diameter: The diameter of the dendritic field was calculated the same way as the diameter for the soma. (iii) Stratification depth: The inner (0\%) and outer (100\%) IPL borders were marked accordingly to the position of blood vessels that demarcate the inner nuclear and ganglion cell layers, respectively. Stratification depth was indicated as the average depth of the dendritic arbor, ramifying through the IPL.





















